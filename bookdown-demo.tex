\documentclass[]{book}
\usepackage{lmodern}
\usepackage{amssymb,amsmath}
\usepackage{ifxetex,ifluatex}
\usepackage{fixltx2e} % provides \textsubscript
\ifnum 0\ifxetex 1\fi\ifluatex 1\fi=0 % if pdftex
  \usepackage[T1]{fontenc}
  \usepackage[utf8]{inputenc}
\else % if luatex or xelatex
  \ifxetex
    \usepackage{mathspec}
  \else
    \usepackage{fontspec}
  \fi
  \defaultfontfeatures{Ligatures=TeX,Scale=MatchLowercase}
\fi
% use upquote if available, for straight quotes in verbatim environments
\IfFileExists{upquote.sty}{\usepackage{upquote}}{}
% use microtype if available
\IfFileExists{microtype.sty}{%
\usepackage{microtype}
\UseMicrotypeSet[protrusion]{basicmath} % disable protrusion for tt fonts
}{}
\usepackage{hyperref}
\hypersetup{unicode=true,
            pdftitle={Programming for Economists},
            pdfauthor={Misja Mikkers and Florian Sniekers},
            pdfborder={0 0 0},
            breaklinks=true}
\urlstyle{same}  % don't use monospace font for urls
\usepackage{natbib}
\bibliographystyle{apalike}
\usepackage{longtable,booktabs}
\usepackage{graphicx,grffile}
\makeatletter
\def\maxwidth{\ifdim\Gin@nat@width>\linewidth\linewidth\else\Gin@nat@width\fi}
\def\maxheight{\ifdim\Gin@nat@height>\textheight\textheight\else\Gin@nat@height\fi}
\makeatother
% Scale images if necessary, so that they will not overflow the page
% margins by default, and it is still possible to overwrite the defaults
% using explicit options in \includegraphics[width, height, ...]{}
\setkeys{Gin}{width=\maxwidth,height=\maxheight,keepaspectratio}
\IfFileExists{parskip.sty}{%
\usepackage{parskip}
}{% else
\setlength{\parindent}{0pt}
\setlength{\parskip}{6pt plus 2pt minus 1pt}
}
\setlength{\emergencystretch}{3em}  % prevent overfull lines
\providecommand{\tightlist}{%
  \setlength{\itemsep}{0pt}\setlength{\parskip}{0pt}}
\setcounter{secnumdepth}{5}
% Redefines (sub)paragraphs to behave more like sections
\ifx\paragraph\undefined\else
\let\oldparagraph\paragraph
\renewcommand{\paragraph}[1]{\oldparagraph{#1}\mbox{}}
\fi
\ifx\subparagraph\undefined\else
\let\oldsubparagraph\subparagraph
\renewcommand{\subparagraph}[1]{\oldsubparagraph{#1}\mbox{}}
\fi

%%% Use protect on footnotes to avoid problems with footnotes in titles
\let\rmarkdownfootnote\footnote%
\def\footnote{\protect\rmarkdownfootnote}

%%% Change title format to be more compact
\usepackage{titling}

% Create subtitle command for use in maketitle
\providecommand{\subtitle}[1]{
  \posttitle{
    \begin{center}\large#1\end{center}
    }
}

\setlength{\droptitle}{-2em}

  \title{Programming for Economists}
    \pretitle{\vspace{\droptitle}\centering\huge}
  \posttitle{\par}
    \author{Misja Mikkers and Florian Sniekers}
    \preauthor{\centering\large\emph}
  \postauthor{\par}
      \predate{\centering\large\emph}
  \postdate{\par}
    \date{2019-11-25}

\usepackage{booktabs}
\usepackage{amsthm}
\makeatletter
\def\thm@space@setup{%
  \thm@preskip=8pt plus 2pt minus 4pt
  \thm@postskip=\thm@preskip
}
\makeatother

\begin{document}
\maketitle

{
\setcounter{tocdepth}{1}
\tableofcontents
}
\chapter{Prerequisites}\label{prerequisites}

No prerequisites required

No prerequisites required.

\chapter{Introduction}\label{intro}

\section{Team}\label{team}

In 2018-2019 this course is taught by:

\begin{itemize}
\tightlist
\item
  Santiago Bohorquez
\item
  Jose Carreno Bustos
\item
  Misja Mikkers
\item
  Marius-Lucian Prisacuta
\item
  Florian Sniekers
\item
  Chayanin Wipusanawan.
\end{itemize}

\section{Datacamp}\label{datacamp}

We are very happy that we partner with datacamp for this course to teach
you both python and R.

Datacamp offers great on-line courses for you to learn R and python.

\section{Important}\label{important}

Things to do if you want to follow this course:

\begin{itemize}
\tightlist
\item
  enroll in datacamp with your \emph{Tilburg University email address}
\item
  go to the russet server
\item
  log into the server and click on the green button ``Start My Server''
\item
  copy the address from the address field in your browser (you need to
  paste this in a webform)
\item
  after you have done this, go to the webform: -and fill in this
  webform.
\item
  note that you need to fill in the webform before \emph{date tobe
  added}
\end{itemize}

If you do not fill in the webform before the deadline, you cannot get a
grade for this course. W e use the webform also to plan the tutorials,
keep an eye on Canvas before your first tutorial.

\section{Questions}\label{questions}

There are no stupid questions, it's stupid not to ask questions.

If you ask us a question by mail, please provide us with the following
information:

\begin{itemize}
\tightlist
\item
  say whether you are an ECO or EBE student
\item
  mention the group number of your tutorial and/or the name of your
  tutorial teacher
\item
  explain your question
\end{itemize}

\chapter{Schedule}\label{schedule}

\section{Before the first lecture}\label{before-the-first-lecture}

\begin{itemize}
\tightlist
\item
  go to the russet server and log in
\item
  create a new notebook by clicking on ``New'' in the (almost) top right
  corner
\item
  choose ``Python 3''
\item
  In the first cell of this notebook, copy-paste the following code:
\end{itemize}

\%\%bash

git clone \url{https://gitlab.uvt.nl/janboone/release.git}

\begin{itemize}
\tightlist
\item
  and press the keys ``SHIFT'' and ``ENTER'' at the same time (``ENTER''
  is sometimes called ``RETURN'')
\item
  the text appears ``Cloning into `release'.''
\item
  this adds a new folder called ``release''; to see this go to ``File''
  -\textgreater{} ``Close and Halt''
\item
  if you click on the folder ``release'', you see that there are three
  sub-folders called ``week1'', ``week2'', ``week3''
\item
  each folder has two notebooks, one called ``assignment'' the other
  ``class''
\item
  the assignment notebook you make before class and we go (quickly) over
  it in class
\item
  the class notebook we do together in class and there will be plenty of
  time for you to answer questions.
\end{itemize}

\section{Schedule}\label{schedule-1}

The schedule for the course is as follows:

\begin{longtable}[]{@{}llll@{}}
\toprule
week & Datacamp & preparation & in class\tabularnewline
\midrule
\endhead
python & & &\tabularnewline
27/1 & Intro to python & bring your laptop to the lecture & Introduction
Lecture\tabularnewline
3/2 & & assignment1.ipynb & class1.ipynb\tabularnewline
10/2 & Intermediate python & &\tabularnewline
17 /2 & & assignment2.ipynb & class2.ipynb\tabularnewline
2/3 & Pandas foundations & &\tabularnewline
9/3 & & assignment3.ipynb & class3.ipynb\tabularnewline
23/3 & midterm python & &\tabularnewline
R & & &\tabularnewline
30/3 & Intro to R & &\tabularnewline
6/4 & & assignment4.ipynb & class4a.ipynb and
class4b.ipynb\tabularnewline
13/4 & Intro to tidyverse & &\tabularnewline
20/4 & & assignment5.ipynb & class5a.ipynb and
class5b.ipynb\tabularnewline
4/5 & & assignment6.ipynb & class6a.ipynb and
class6b.ipynb\tabularnewline
18/5 & reserve & &\tabularnewline
TBA & exam R & &\tabularnewline
TBA & resit python and R & &\tabularnewline
\bottomrule
\end{longtable}

Note that the exam dates are not currently planned. Also at this moment
the exam times are not known.

For the first lecture it is useful to bring your laptop if you have one.
We will show you how to start the server and evaluate python in a
jupyter notebook.

Also in the week of the first lecture, do the Datacamp course with the
introduction to python.

After the lecture, we have alternating weeks of classes and Datacamp.
Before you come to class, you do the assignment notebook at home. In
class we will do the class notebook together. The idea of this is that
you can practice python while we are around to quickly answer your
questions, help you with errors etc.

Make sure that you have your notebooks complete before the exam. We will
not publish solutions to the notebooks. But if you have your completed
notebooks on the russet server, you are allowed to use copy/paste at the
exam.

In the week without a class, you do the Datacamp lecture at home. With
Datacamp you learn the python syntax (what code to type). In class we
teach you how to use python to solve economic problems.

\section{Teaching philosophy}\label{teaching-philosophy}

We teach python/R in a way that may seem counter-intuitive to you at
first sight. Our starting point is that you are not doing an
engineering/IT degree; you are an economist. Hence, we are interested in
economic applications; not so much in the details of the python/R
language.

Of course, you need to learn the basics of the language, things like
data-types, control structures etc. In other words, you need to know
what a list is, a dictionary, a numpy array, a pandas dataframe, how to
define functions, use list comprehension, if-then-else structures etc.

This ``technical'' part you mainly learn via Datacamp. There they teach
you the details of the language such that you get the syntax right. By
doing their exercises you learn to type python, recognize error messages
(when you make mistakes).

In class we then focus on applying these tools to economic problems. In
class we also type python and we also make mistakes which generate
errors. So there is an overlap with the Datacamp lectures, but the focus
in class is different. Further, in economics we are interested in
optimization (we tend to maximize profits, utility etc.) and solving
equations (specifically, solving for fixed points which give us
equilibria). For this we use numpy and scipy, which is not covered much
at the Datacamp courses that we do.

Many concepts you will first see at Datacamp and then we apply them in
class. Sometimes it will be the other way around: we used something in
class and you learn more details about it at Datacamp. This is perfectly
fine. However, it is important that you keep up-to-date with Datacamp
otherwise you are going to get lost. Also programming is something that
you need to practice. You can do the same the Datacamp two or three
times. Also the notebooks that we do in class, you can play around with
these. Plot different functions, solve equations for different parameter
values etc. Just looking at the answers that we give you in class will
not help you to learn to use python/R.

Finally, we urge you to use google (or other search engines like
DuckDuckGo) and stackoverflow with your assignments. Some students find
this weird at the beginning: should we not teach you everything that you
need to know? The answer is no for a number of reasons. First, even
professional programmers use google and stackoverflow all the time. If
you are on Quora; see this post and this one. Second, python and R are
open source and lots of people work with it. If you encounter a problem,
chances are that someone else had the same problem and knows the
solution to it. There is not need to ``invent the wheel''. Use the
resources available to you. If you copy a lot of code, you should add a
reference. Finally, because python and R are open source, they develop
rapidly. The things that we teach you now, will be obsolete in a couple
of years time. Hence, you need to be able to find your way around also
in 10 years time. To start practicing this, use google now.

The only warning here is: at the exam you will not have access to the
whole internet. So, also make sure that you can find help in the jupyter
notebook. We will practice this in class.

\chapter{Important}\label{important-1}

\emph{Things to do if you want to follow this course:}

\begin{itemize}
\tightlist
\item
  enroll in datacamp with your Tilburg University email address
\item
  go to the russet server
\item
  log into the server and click on the green button ``Start My Server''
\item
  copy the address from the address field in your browser (you need to
  paste this in a webform)
\item
  after you have done this, go to the webform:
\item
  and fill in this webform.
\item
  note that you need to fill in the webform before \emph{date to be
  added}
\item
  if you do not fill in the webform before the deadline, you cannot get
  a grade for this course
\item
  we use the webform also to plan the tutorials, keep an eye on Canvas
  before your first tutorial
\end{itemize}

\chapter{Information first Lecture}\label{information-first-lecture}

\section{Introduction Programming}\label{introduction-programming}

Misja Mikkers \& Florian Sniekers

\section{Table of Contents}\label{table-of-contents}

\begin{itemize}
\item
  Introduction
\item
  markdown
\item
  Second part of lecture
\end{itemize}

\section{Introduction}\label{introduction}

\begin{itemize}
\tightlist
\item
  Don't panic
\item
  this is a programming course
\item
  we know that many of you are not too keen on computers (beyond MS
  Office)
\item
  this will be a gentle introduction to open source software
\item
  it will not become too sophisticated
\item
  it is meant for everyone to understand
\item
  especially, if you never did any programming before
\item
  Why this course?
\item
  mainly to teach you to use your computer better
\item
  to be able to use open source (``free'') software
\item
  to solve problems together with readable documentation on how you
  solved it (``reproducible research'')
\item
  on this last point, office products like excel score rather badly
\item
  you will use R and python in courses in the years to come
\end{itemize}

\subsection{Who teaches this course?}\label{who-teaches-this-course}

In 2019-2020 this course is taught by:

Misja Mikkers Florian Sniekers \emph{add TA's}

\subsection{How do we teach this
course?}\label{how-do-we-teach-this-course}

\begin{itemize}
\tightlist
\item
  on line lectures on Datacamp
\item
  tutorials: with plenty of time to ask questions
\item
  there are a number of ``regular'' tutorials and one in the computer
  lab
\item
  if you do not have a laptop, attend the tutorial in the computer lab
\item
  if you do have a laptop, attend your regular tutorial group
\item
  no need to attend both!
\item
  we may drop some tutorial groups, so check Canvas!
\item
  schedule can be found in the chapter schedule on this website
\item
  we can track your progress on datacamp
\item
  assignment notebooks to be made before the class
\item
  class notebooks that we do together in class (to allow you to ask
  questions)
\end{itemize}

\subsection{Information about the
course}\label{information-about-the-course}

\begin{itemize}
\item
  all information about the course can be found on this website
\item
  pay attention to: -the schedule: explaining when you need to do what
  -the rules for the exam explaining how the exam works and a practice
  exam
\item
  Your grade -There are two separate ways to earn your grade
\item
  regular route: -midterm on python -end of semester exam on R
\item
  resit -exam on python and R combined -you cannot use grades from one
  route for the other one.
\end{itemize}

\subsection{Exam}\label{exam}

\begin{itemize}
\tightlist
\item
  for more information see the exam chapter
\item
  check the instructions for the exam
\item
  do not open your exam file after you have finished
\item
  if you do, your exam will not be graded (even if you did not change
  anything)
\item
  at the exam you can freely copy and paste from the assignments we did
  in class
\item
  we will not post answers to the assignments
\item
  make sure you attend the tutorials and pay attention in class!
\end{itemize}

\subsection{Datacamp}\label{datacamp-1}

\begin{itemize}
\tightlist
\item
  you need to sign up for Datacamp!
\item
  for details see the chapter Important
\item
  note the deadline for filling in the webform!
\item
  if you miss the deadline, you will have to pay for premium content on
  Datacamp yourself
\end{itemize}

\subsection{markdown}\label{markdown}

\begin{itemize}
\item
  syntax
\item
  markdown allows you to create structure in a simple way
\item
  examples are: \texttt{\#\ this\ is\ a\ heading}

  \texttt{\#\#\ subheading}

  \texttt{*\ first\ bullet} \texttt{*\ second\ bullet}

  \texttt{{[}link\ text{]}(actual\ link,\ e.g.\ http://www.etc)}

  \texttt{!{[}Alt\ text\ for\ image{]}(/path/to/img.jpg\ "Optional\ title")}
\item
  look on the web for other syntax like footnotes etc.
\item
  equations you can type in latex
\item
  latex is great word processing software for now, we only need it to
  write math you can guess what the following will do:
\end{itemize}

\texttt{\$x\^{}2\$,\ \$\textbackslash{}beta\$,\ \$\textbackslash{}sqrt\{9\}\$,\ \$\textbackslash{}frac\{1\}\{2\}\$,\ \$\textbackslash{}bar\ x\$}

\texttt{\textbackslash{}begin\{equation\}}

\texttt{a\^{}2\ +\ b\^{}2\ =\ c\^{}2}

\texttt{\textbackslash{}end\{equation\}}

\begin{itemize}
\tightlist
\item
  if you need something, just google; e.g. ``google latex phi'' or
  ``google latex empty set'' etc.
\item
  and try it out in the jupyter notebook
\end{itemize}

\section{Second part of lecture}\label{second-part-of-lecture}

\begin{itemize}
\item
  go to the server and start a jupyter notebook
\item
  link to the server address to copy/paste in the google form
\item
  importing the python material (this is also the way you will import
  your exam)
\item
  evaluating cells
\item
  you can choose python/R kernel
\item
  getting help: ? and TAB
\item
  code vs.~markdown cell
\item
  type some latex
\end{itemize}

\subsection{before you leave}\label{before-you-leave}

do the steps under the chapter Important

\chapter{Exam}\label{exam-1}

\section{Grade}\label{grade}

Your grade is either determined by:

\begin{itemize}
\tightlist
\item
  the midterm exam which is python only (50\%)
\item
  the exam will be R only (50\%)
\end{itemize}

or by

the resit which is based on both python and R (100\%)

Each exam lasts 3 hours. You cannot combine the resit with the midterm
etc.

\section{Useful to know}\label{useful-to-know}

The questions that we ask in the exam are based on the notebooks that we
discuss in class. Hence make sure that you have these ready before the
exam. You are allowed to use copy-paste out of these notebooks.

During the semester, you can use google to find information on
functions, error messages etc. However, during the exam you can only
access a limited number of pages.

In particular, during the exam you work in a special exam environment on
TiU computers. We have asked IT to whitelist the following websites:

\begin{itemize}
\tightlist
\item
  the russet server (where you will do your exam)
\item
  \href{gitlab.uvt.nl}{gitlab} (where you will import your exam from)
\item
  Canvas (where we will give you the command to import the exam)
\item
  \href{https://stackoverflow.com/}{stackoverflow}
\item
  \href{https://www.python.org/}{python.org}
\item
  \href{http://www.numpy.org/}{numpy}
\item
  \href{https://www.scipy.org/}{scipy}
\item
  \href{https://pandas.pydata.org/}{pandas}
\item
  \href{https://www.datacamp.com/home}{datacamp}
\item
  \href{http://www.cookbook-r.com/}{cookbook}
\item
  \href{https://dplyr.tidyverse.org/}{tidyverse.org}
\item
  \href{https://rpubs.com/}{rpubs}
\item
  \href{https://www.rstudio.com/}{rstudio}
\item
  \href{https://r4ds.had.co.nz/}{r4ds}
\item
  \href{https://www.rdocumentation.org/}{rdocumentation}
\end{itemize}

\section{Exam procedure}\label{exam-procedure}

\begin{itemize}
\tightlist
\item
  we will post the exam both on gitlab. You will get instructions how to
  get the exam on the Russet server.r
\item
  finishing your exam

  \begin{itemize}
  \tightlist
  \item
    make sure that we can easily see which notebook is your exam
  \item
    that is, do not rename the exam notebook (so that we do not know
    which notebook it is)
  \item
    do not have 5 different versions of the exam notebook; we will then
    choose one at random and grade this one
  \item
    after you finish the exam, do not re-open the notebook again: we can
    see the last time the notebook was opened. If this is after you left
    the exam room, we can see this and will nullify your exam.
  \end{itemize}
\end{itemize}

\bibliography{book.bib,packages.bib}


\end{document}
